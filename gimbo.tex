\documentclass[14pt]{article}

\begin{document}

\begin{center}\begin{small} MAKERERE UNIVERSITY \end{small}\end{center}

\begin{center}\begin{small} COLLEGE OF COMPUTING AND INFORMATION SCIENCES \end{small}\end{center}

\begin{flushleft}\begin{small}NAME:  GIMBO IMELDA THRESA \end{small}\end{flushleft}

\begin{flushleft}\begin{small} REG NO 13/U/5237/PS \end{small}\end{flushleft}

\begin{flushleft}\begin{small} STUD NO: 213012723 \end{small}\end{flushleft}

\author{G.IMELDA}

\title{A REPORT ON COOKING BY STUDENTS IN HOSTELS}

\maketitle

\section{INTRODUCTION}
I have a very severe cooking problem, which has developed recently along with the long period stay in hostels during campus time.  Our hostel management department proposed having an excellent system of some ladies providing food for sale to students to solve the problem, but to me that was not satisfying.          

\section{BACKGROUND}
A few years cooking in hostels was not a problem, though this has become a serious problem now with the large bills of electricity, arson, damage and frustration resulting. Hostel managements decided to implement a system for students where certain ladies come to hostels providing food for sale to students.
This to me as a student in hostel was not a satisfying solution because some of us like eating the food cooked by ourselves. 

\section{DISCUSSION}
The cooking system at hostel was a very big challenge because I personally was used to the restaurant food but by the time I happened to find an insect in my bean soup, I gave up on buying food because that really showed how food was not carefully prepared which could easily bring about disorders in the body and infectious diseases such as diarrhea that could result to hospital treatments causing spending of the little money that was not ready for that due to lack of good planning.
The result that I faced in the buying food forced me to start cooking for myself in the hostel, I had to learn how to cook food and I can actually do it better than any other thing.  Cooking is now something part of me, it’s a hobby now. This involves lighting a charcoal stove or a fuel stove since using electricity to cook in hostels was highly prohibited. It also involves the use of different ingredients like cooking oil, onions, green peppers, red peppers, carrots, tomatoes, rosemary, hot meal, curry powders, salt and many other things. Some of these ingredients like onions, carrots, green peppers, red peppers and tomatoes need to be fast chopped before being used for cooking.  Different tools are also used for example; spoons, forks, knife, chopping boards, saucepans, plates and a mingling stick. I can cook different foods for example Irish mixed with spaghetti, Irish mixed with egg, spaghetti mixed with rice, Irish mixed with meat and many more dishes. There are also some challenges I face during my cooking times like, putting too much salt in my food sometimes I cooked, other days I added a lot of water and sometimes I made the food burn and it turned black, but slow by slow I kept on improving on my cooking skills that gave me advantages for cooking for myself whenever I wanted to eat food.
This really helped me to save money for myself that helped me do other different thing like shopping for myself certain things like clothes, shoes and bags. It also improved my standards of living, I prepared food once a day that I fed on like two times with friends who would be around and also plan for the next day, I learnt how to stop feeding on junks and have a good feeding timetable because now I was able to prepare food myself and gave me skills of learning more how to make good food.

\section{CONCLUSION}
This report has identified the current situation on the feeding of students in hostels. Some students just buy food from restaurant ladies who come around at hostels while others like me just cook the meals for themselves.  The decision of how to cater for meals depends on the student’s choice of either to purchase already cooked food or to prepare the food for themselves. The idea of students preparing meals for themselves is dependent on certain impacts like untrustworthy of the ladies already prepared food, their hygiene and sanitation.  The uncertain long period of time they take to deliver the food and sometimes the very poor customer care and communication skills. Cooking has improved significantly in these recent years as 80 percent of the students prepare meals for themselves. The impact of this improvement is expected to increase the number of student preparing meals for themselves and also the permission of certain electricity gadgets like percolators, rice cookers, blenders and fridges. In relation to this, cooking can be expected to continue with other forms of activities like managing businesses by students for examples certain students selling foods like matooke, rice, sweet potatoes, and Irish potatoes to other students who cook.

\end{document}

